\chapter{はじめに}
\label{chap:00-preface}

このたびは、「やる夫で学ぶ{-}Redux{-} Redux Toolkitは簡単だお・・・」を、手にしていただき誠にありがとうございます。
\\[0pt]
\\[0pt]
私は15年近くVisual Studio上でC\#を使ってUIのあるプログラムを書いていました。
昨今は、同一コードでマルチプラットフォーム上で動作するアプリケーションに向かっています。
\\[0pt]
今では、Microsoft .NETがマルチプラットフォームで動作するようになり、UIなども充実し始めています。
\\[0pt]
あるとき、あるプロジェクトでは、「Windows、Mac上にて同一UIで動作」が要求されました。
そのために選択したのが、ブラウザ上で動作するWebアプリケーションです。
\\[0pt]
さらに、Webアプリケーションではローカルファイルにアクセスできないため、Electronを使用することで、
Webアプリケーションをマルチプラットフォームで動作するアプリケーションにしました。

当時は、WebアプリケーションのフレームワークとしてAngularとReactが候補に挙がりましたが、
Reactの方が学習コストが低いと言われていたため、Reactを学びました。

本書は、私がReactやReduxを使い始めたときに「こんな本があれば良かったのに・・・」を目指して書きました。
React,Reduxの初学者から中級の方のお役に立てれば幸いです。

そして、私が一番欲しかった「常に最新の情報」を掲載していきたいと思っています。

React、Reduxの解説書やチュートリアルは、書籍・インターネット上にたくさんあります。
しかし、つぎつぎと新機能が追加され本家以外の情報は、あっという間に古くなってしまいます。
この書籍は、掲載情報を最新にするために電子書籍のみとし、古くなった情報は随時更新していきたいと思っています。

お気付き点がございましたら、Guthub上へお寄せください。

本書は、フロントエンド開発で使用されている\\[0pt]
* react
* redux(redux{-}toolkit)

を習得するために、開発環境の作成(つまりゼロ)から始め、チュートリアルの定番のToDoリストを作成します。

すでに開発環境を整えている方は、第1章を飛ばしてもかまいません。

また、作成するToDoアプリケーションはGutHubに公開していますが、こちらも最新の情報に更新していくつもりです。
GutHubでは、章毎に別ブランチにしてありますので、写経がメンドウな方は章に対応したブランチを使ってください。

git、GitHubの使い方については、本書では取り扱いません。

本書は、チュートリアルによくあるToDoリストを\\[0pt]

\begin{starterenumerate}
\item Reduxなしで作成
\item Reduxを導入
\item Redux Toolkitを導入
\end{starterenumerate}

と、書き換えていくことでReduxを用いた「状態管理(アプリケーション全体でのデータ)」を
理解してもらえるようになっています。

 create{-}react{-}appを使用して、ゼロからの手順を解説しています。写経がメンドウな人は、
GitHubにあるコードを使ってください。

必要なものは、すべて無料でそろえることができる以下のものです。これらが何なのか、そして、インストール方法は
第1章にて解説しています。
 * Node.js
 * yarn(npmを使われる方は不要)
 * Microsoft Visual Studio code
 * Google Chrome

それでは、始めていきましょう。
